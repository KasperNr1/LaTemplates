\documentclass[
    fontsize=11pt,
    paper=a4,
    DIN,                 % DIN 5008 Layout
    parskip=half,        % Halbe Leerzeile zwischen Absätzen
    enlargefirstpage,    % Mehr Platz auf der ersten Seite
    fromalign=right,     % Absender rechts oben
    fromphone=true,      % Telefonnummer anzeigen
    fromemail=true,      % Email anzeigen
    fromurl=false,       % Webseite anzeigen (optional)
    fromlogo=false,      % Logo anzeigen (optional)
    subject=titled,      % Betreff hervorheben
]{scrlttr2}

% --- DEUTSCHE SPRACHE & ZEICHENSATZ ---
\usepackage[utf8]{inputenc}
\usepackage[T1]{fontenc}
\usepackage[ngerman]{babel} % Wichtig für "z.B.", Datum, Trennung
\usepackage{lmodern}        % Scharfe Schriftart
\usepackage{eurosym}        % Das offizielle € Zeichen (\euro)

% --- TABELLEN & BERECHNUNG ---
\usepackage{booktabs}       % Schönere Linien (\toprule etc.)
\usepackage{spreadtab}      % Für die automatischen Berechnungen
\usepackage{numprint}       % Zahlenformatierung

% Einstellungen für deutsche Zahlen (Komma statt Punkt)
\STsetdecimalseparator{,}
\npthousandsep{.}
\npdecimalsign{,}

% --- DEINE ABSENDERDATEN ---
\setkomavar{fromname}{Max Mustermann}
\setkomavar{fromaddress}{Musterstraße 1\\12345 Musterstadt}
\setkomavar{fromphone}{+49 123 456 789}
\setkomavar{fromemail}{max@mustermann.de}
\setkomavar{place}{Musterstadt} % Ort vor dem Datum
\setkomavar{signature}{Max Mustermann} % Unterschrift

%IBAN & Bankdaten für die Fußzeile
\setkomavar{firstfoot}{
    \footnotesize \color{gray}
    \centering
    \begin{tabular}{c c c}
         \usekomavar{fromname} & IBAN: DE12 3456 7890 1234 56 & Steuernummer: 12/345/67890 \\
         \usekomavar{fromaddress} & BIC: XXXXXXXXXXX & Amtsgericht Musterstadt
    \end{tabular}
}

\begin{document}

% --- EMPFÄNGER DATEN ---
\begin{letter}{
    Beispielkunde GmbH \\
    Einkaufsabteilung \\
    Industriestraße 55 \\
    \\
    54321 Kundenstadt
}

% --- RECHNUNGSDATEN ---
\setkomavar{subject}{Rechnung Nr. 2025-001}
\setkomavar{customer}{KD-1005} % Kundennummer (erscheint im Infoblock)
\setkomavar{date}{\today}

\opening{Sehr geehrte Damen und Herren,}

vielen Dank für Ihren Auftrag. Hiermit stelle ich Ihnen folgende Leistungen in Rechnung:

\vspace{1em}

% --- BERECHNUNGSTABELLE ---
% WICHTIG: Eingabe der Preise mit PUNKT (45.00), Ausgabe erfolgt als Komma
\begin{spreadtab}{{tabular}{p{8cm} r r r}}
    \toprule
    \textbf{Beschreibung} & \textbf{Menge} & \textbf{Einzel (\euro)} & \textbf{Gesamt (\euro)} \\
    \midrule
    \vspace{2pt} \\
    
    % Position 1
    @ IT-Support und Wartung & 5 & 60.00 & :={[-2,0]*[-1,0]} \\
    
    % Position 2
    @ Implementierung Backend & 12.5 & 85.00 & :={[-2,0]*[-1,0]} \\
    
    % Position 3
    @ Server-Setup Pauschale & 1 & 150.00 & :={[-2,0]*[-1,0]} \\
    
    \midrule
    
    % --- SUMMEN ---
    @ \multicolumn{3}{r}{Zwischensumme Netto:} & :={sum(d3:d6)} \\
    
    % 19% MwSt (0.19). Für Kleinunternehmer Zeile löschen!
    @ \multicolumn{3}{r}{Umsatzsteuer (19\%):} & :={[-1,0]*0.19} \\
    
    @ \multicolumn{3}{r}{\textbf{Gesamtbetrag:}} & \textbf{:={[-2,0]+[-1,0]}} \\
    
    \bottomrule
\end{spreadtab}

\vspace{1.5em}

Bitte überweisen Sie den Gesamtbetrag von \textbf{\STtag{cell(d9)} \euro} innerhalb von 14 Tagen auf das unten genannte Konto.
Das Leistungsdatum entspricht dem Rechnungsdatum.

\closing{Mit freundlichen Grüßen}

\end{letter}
\end{document}