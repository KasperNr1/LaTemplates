% LaTeX Simple Invoice Template
% Original by Amy Fare | amyfare.ca
% Adapted for German/Automatic Calculation by Gemini

\documentclass{letter}
\usepackage[utf8]{inputenc}
\usepackage[T1]{fontenc}      % Wichtig für Umlaute
\usepackage[ngerman]{babel}   % Deutsche Spracheinstellungen
\usepackage[colorlinks]{hyperref}
\usepackage[left=1in,top=1in,right=1in,bottom=1in]{geometry} 
\usepackage{graphicx}
\usepackage{tabularx}
\usepackage{multirow}
\usepackage{ragged2e}
\usepackage{hhline}
\usepackage{array}
\usepackage{eurosym}          % Für das € Zeichen
\usepackage{spreadtab}        % Für automatische Berechnungen
\usepackage{numprint}         % Zahlenformatierung (1.000,00)

\hypersetup{
    urlcolor=blue
}

% Einstellungen für deutsche Zahlenformatierung
% \STsetdecimalseparator{,} 
\npthousandsep{.}
\npdecimalsign{,}

% Definition für rechtsbündige Spalten in TabularX
\newcolumntype{R}{>{\raggedleft\arraybackslash}X}

\begin{document}
    
\thispagestyle{empty}

% --- KOPFZEILE ---
% Hier Logo und deine Firmeninfos anpassen
\begin{tabularx}{\textwidth}{l X l}
   \hspace{-8pt} \multirow{5}{*}{\includegraphics[width=2.5cm]{example-image}} & \textbf{Dein Firmenname} & \hskip12pt\multirow{5}{*}{\begin{tabular}{r}\footnotesize\bf RECHNUNG \\[-0.8ex] \footnotesize NR. 2025-001 \\[-0.4ex] \footnotesize\bf DATUM \\[-0.8ex] \footnotesize \today \\[-0.4ex] \footnotesize\bf FÄLLIG \\[-0.8ex] \footnotesize 14 TAGE NACH ERHALT \end{tabular}}\hspace{-6pt} \\
   & Dein Name & \\
   & deine-website.de & \\
   & +49 123 456789 & \\
   & mail@beispiel.de & \\
\end{tabularx} 

\vspace{1 cm}

\textbf{RECHNUNGSEMPFÄNGER}

\Large\textbf{Musterkunde GmbH}\normalsize\\
Musterstraße 55\\
12345 Musterstadt

\vspace{1 cm}

% --- TABELLE MIT BERECHNUNG ---
% Spalten: Beschreibung (X), Einzelpreis (r), Menge (c), Gesamt (r)
\begin{spreadtab}{{tabularx}{\linewidth}{X r c r}}
    \hline
    & & & \\[0.25ex]
    @ \textbf{Leistungsbeschreibung} & @ \textbf{Einzel (\euro)} & @ \textbf{Menge} & @ \textbf{Gesamt (\euro)} \\[1ex] \hline
    & & & \\
    
    % --- POSITIONEN ---
    % WICHTIG: Preise mit PUNKT eingeben (z.B. 60.00). LaTeX macht daraus Kommas.
    % Formel :={[-2,0]*[-1,0]} bedeutet: Spalte Preis * Spalte Menge
    
    @ IT-Support & 60.00 & 2 & :={[-2,0]*[-1,0]} \\[1.5ex]
    
    @ Software-Installation & 25.50 & 4 & :={[-2,0]*[-1,0]} \\[1.5ex]
    
    @ Anfahrtspauschale & 15.00 & 1 & :={[-2,0]*[-1,0]} \\[1.5ex]
    
    % --- SUMMEN ---
    \hline
    & & & \\
    @ & @ \multicolumn{2}{r}{Netto:} & :={sum(d4:d8)} \\[1ex]
    
    % MwSt (0.19). Falls Kleinunternehmer: Zeile löschen oder Faktor auf 0 setzen
    @ & @ \multicolumn{2}{r}{Umsatzsteuer (19\%):} & :={[-1,0]*0.19} \\[1ex]
    \hhline{~~~=}
    
    @ & @ \multicolumn{2}{r}{\textbf{Gesamtbetrag:}} & \textbf{:={[-2,0]+[-1,0]}} \\[1ex]
    \hhline{~~~=}
\end{spreadtab}

\vspace{1.5 cm}

\Large\textbf{Zahlungsinformationen}\normalsize

\vspace{0.2 cm}

Bitte überweisen Sie den Betrag innerhalb von 14 Tagen auf folgendes Konto:

\vspace{0.3 cm}

\begin{tabular}{ll}
    \textbf{Bank:} & Musterbank \\
    \textbf{IBAN:} & DE00 1234 5678 9012 3456 78 \\
    \textbf{BIC:}  & XXXXXXXXXXX \\
    \textbf{Verwendungszweck:} & Rechnungsnr. 2025-001
\end{tabular}

\vspace{1cm}
\footnotesize
\color{gray}
Steuernummer: 12/345/67890 $\cdot$ Amtsgericht Musterstadt

\end{document}