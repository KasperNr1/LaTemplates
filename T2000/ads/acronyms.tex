\addchap{Abkürzungsverzeichnis}
%nur verwendete Akronyme werden letztlich im Abkürzungsverzeichnis des Dokuments angezeigt
%Verwendung: 
%		\ac{Abk.}   --> fügt die Abkürzung ein, beim ersten Aufruf wird zusätzlich automatisch die ausgeschriebene Version davor eingefügt bzw. in einer Fußnote (hierfür muss in header.tex \usepackage[printonlyused,footnote]{acronym} stehen) dargestellt
%		\acs{Abk.}   -->  fügt die Abkürzung ein
%		\acf{Abk.}   --> fügt die Abkürzung UND die Erklärung ein
%		\acl{Abk.}   --> fügt nur die Erklärung ein
%		\acp{Abk.}  --> gibt Plural aus (angefügtes 's'); das zusätzliche 'p' funktioniert auch bei obigen Befehlen
%	siehe auch: http://golatex.de/wiki/%5Cacronym
%	

% Setzt den Horizontalen Abstand zwischen Acronym und Erklärung so, dass auch ein Ac. mit 6 Zeichen passt.
\begin{acronym}[012345] % Nur die Menge der Zeichen ist für den Parameter relevant   
\setlength{\itemsep}{-0.5\parsep} % Zeilenabstand zwischen Einträgen

\acro{WIP}{Work in Progress}
\end{acronym}