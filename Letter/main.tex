\documentclass[a4paper]{letter}
\usepackage[ngerman]{babel}
\usepackage[businessenvelope]{envlab}
\usepackage{hyperref}

% Absender Adresse
\address{Straße Nr. 12 \\ 12345 Ortschaft}
\location{Ort des Erstellers}
\telephone{0112497-1274}

\signature{Der Author}

\begin{document}
% \rmfamily  %% Serif-Font 
% \sffamily  %% Sans-Serif-Font

% Empfänger Adresse
\begin{letter}{Professor \\ Dick \& Doof \\ Musterstraße 0815 \\ 12345 Stadtname}
\opening{Sehr geehrter Leser}

Ich hoffe der Brief entspricht Ihren Ansprüchen. Mir ist bewusst, dass diese teils sehr hoch sind, jedoch müsst mit diesem Dokument eine große Vielfalt des offizielleren Schriftverkehrs geregelt sein. Ich bin zur Zeit in der Findungsphase auf der Suche nach dem Perfekten Vorlagen-Ordner. 

Am heutigen Tage soll er jedoch um einen weiteren Eintrag gewachsen sein. Viel Vergnügen bei der Benutzung. 

Hoffentlich kommt er auch in der Zukunft gut an und erspart die tragischen Umstände eines Office--365 Abos.

Hier ist ein tolles Integral um Style--Punkte zu sammeln und den Inhalt zu füllen:

\begin{equation}
    \int_{0}^{\infty} e^{x} = \infty 
\end{equation}

\closing{Mit dem Ihnen gebührlichen Respekt, der sehr lang ist weil dir tatsächlich etwas Respekt zusteht.}

\vspace{2cm}
\hrule
\ps

P.S. You can find the full text of GFDL license at
\url{http://www.gnu.org/copyleft/fdl.html}.

Ich hoffe der Brief entspricht Ihren Ansprüchen. Mir ist bewusst, dass diese teils sehr hoch sind, jedoch müsst mit diesem Dokument eine große Vielfalt des offizielleren Schriftverkehrs geregelt sein. Ich bin zur Zeit in der Findungsphase auf der Suche nach dem Perfekten Vorlagen-Ordner. 

% Um Anhänge aufzuzählen
% \encl{Copyright permission form}

\end{letter}
\end{document}