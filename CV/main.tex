%%%%%%%%%%%%%%%%%%%%%%%%%%%%%%%%%%%%%%%%%%%%%%%%%%%%%%%%%%%%%%%%%%%%%%%%%%%%%%%%
%
% Professionelle Lebenslauf-Vorlage mit der 'moderncv'-Klasse
%
%%%%%%%%%%%%%%%%%%%%%%%%%%%%%%%%%%%%%%%%%%%%%%%%%%%%%%%%%%%%%%%%%%%%%%%%%%%%%%%%

% \documentclass[11pt, a4paper, roman]{moderncv} % 'roman' für Serifen-Schrift
\documentclass[11pt, a4paper]{moderncv}

% --- THEME & FARBE ---
% Stile: 'classic', 'casual' (Standard), 'oldstyle', 'banking'
\moderncvstyle{classic} 
% Farben: 'blue' (Standard), 'orange', 'green', 'red', 'purple', 'grey', 'black'
\moderncvcolor{blue} 

\usepackage[utf8]{inputenc}
\usepackage[ngerman]{babel}
\usepackage[T1]{fontenc}

% --- SEITENRÄNDER ---
\usepackage[scale=0.78]{geometry} % Passt die Ränder an (0.75 = mehr Rand, 0.8 = weniger Rand)

% --- PERSÖNLICHE DATEN ---
% Diese Befehle füllen die Kopfzeile und den Titel
\name{Max}{Mustermann}
\title{Lebenslauf} % Optional, erscheint unter dem Namen

% \address{Straße Nr.}{Postleitzahl Stadt}{Land}
\address{Musterstraße 123}{12345 Musterstadt}{} % Land kann leer gelassen werden

% \phone[Typ]{Nummer} (Typen: mobile, fixed)
\phone[mobile]{+49 123 4567890}
\email{max.mustermann@email.de}
\homepage{www.max-mustermann.de} % z.B. für Portfolio
\social[github]{github.com/max-mustermann}

% Optional: Foto
% \photo[64pt][0.4pt]{bilder/profilbild.jpg} % [Größe][Rahmen]

% --- DOKUMENTEN-START ---
\begin{document}

% --- TITEL ERSTELLEN ---
% Erstellt den kompletten Kopfbereich mit allen \name, \address etc. Daten
\makecvtitle

% =======================================================
% PROFIL (Optional)
% =======================================================
% \section* (unnummeriert) wird hier für den Fließtext-Abschnitt verwendet
\section*{Profil}
\cvitem{}{\small Motivierter und lösungsorientierter Data Scientist mit 5 Jahren Erfahrung in der Entwicklung von Machine-Learning-Modellen. Spezialisiert auf die Bereiche Natural Language Processing und Python. Suche nach einer neuen Herausforderung in einem datengetriebenen Team.}


% =======================================================
% BERUFSERFAHRUNG
% =======================================================
\section{Berufserfahrung}

% \cventry{Zeitraum}{Position}{Arbeitgeber}{Ort}{Note (optional)}{Beschreibung}
\cventry{01/2022 - Heute}{Senior Data Scientist}{Data Solutions GmbH}{Berlin}{}{
\begin{itemize}
    \item Entwicklung und Implementierung von NLP-Modellen zur Textklassifizierung.
    \item Leitung eines 3-köpfigen Junior-Teams und Mentoring.
    \item Erstellung von Dashboards (Tableau) für das Manage
    ment.
\end{itemize}
}

\cventry{06/2019 - 12/2021}{Data Scientist}{Musterfirma AG}{Hamburg}{}{
\begin{itemize}
    \item Analyse von Kunden-Churn-Daten und Entwicklung eines Vorhersagemodells.
    \item Durchführung von A/B-Tests zur Optimierung von Marketingkampagnen.
    \item Datenaufbereitung und -bereinigung von SQL-Datenbanken.
\end{itemize}
}

\cventry{03/2018 - 05/2019}{Werkstudent Business Intelligence}{Beispiel-Konzern}{München}{}{
\begin{itemize}
    \item Erstellung von wöchentlichen Reports und Ad-hoc-Analysen mit SQL und Excel.
\end{itemize}
}

% =======================================================
% AUSBILDUNG
% =======================================================
\section{Ausbildung}

\cventry{10/2017 - 05/2019}{Master of Science: Data Science}{Technische Universität Musterstadt}{Musterstadt}{Note 1,3}{
    \textit{Masterarbeit: "Sentiment-Analyse von Produktbewertungen"} \newline
    Schwerpunkte: Machine Learning, Deep Learning, Big Data Technologien
}

\cventry{10/2014 - 09/2017}{Bachelor of Science: Wirtschaftsinformatik}{Universität Beispiel}{Beispielort}{Note 1,7}{
    \textit{Bachelorarbeit: "Entwicklung einer Web-App mit Django"}
}

% =======================================================
% KENNTNISSE & FÄHIGKEITEN
% =======================================================
\section{Kenntnisse}

% \cvitem{Kategorie}{Inhalt}
\cvitem{Sprachen}{Deutsch (Muttersprache), Englisch (Verhandlungssicher C1), Spanisch (Grundkenntnisse A2)}

\cvitem{Programmie\-rung}{Python (Pandas, Scikit-learn, TensorFlow), SQL, R, Git}

\cvitem{Software}{Tableau, Power BI, Docker, AWS (S3, SageMaker), MS Office}

% --- Alternativ: \cvlistdoubleitem für zwei Spalten ---
% \cvlistdoubleitem{Python}{Pandas}
% \cvlistdoubleitem{SQL}{R}

% =======================================================
% INTERESSEN (Optional)
% =======================================================
\section{Interessen \& Ehrenamt}
\cvitem{Ehrenamt}{IT-Support im lokalen Gemeindezentrum}
\cvitem{Interessen}{Bergwandern, Fotografie}


\end{document}