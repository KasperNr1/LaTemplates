\documentclass[%
%	draft,				% Einkommentieren um Satzfehler zu sehen
	final,				% Einkommentieren für Abgabe (Kommentare werden ausgeblendet etc.)
	pdftex,
	oneside,			% Einseitiger Druck.
	12pt,				% Schriftgroesse
	parskip=half,		% Halbe Zeile Abstand zwischen Absätzen.
	headheight = 0pt,	% Höhe der Kopfzeile
	footheight = 25pt,	% Höhe der Fusszeile
	abstracton,		% Abstract Überschriften
	headings=normal,	% Größe der Überschrifen = bigheadings, normalheadings, smallheadings
	DIV=calc,		% Satzspiegel berechnen
	BCOR=8mm,		% Bindekorrektur links: 8mm
	headinclude=false,	% Kopfzeile nicht in den Satzspiegel einbeziehen
	footinclude=false,	% Fußzeile nicht in den Satzspiegel einbeziehen
	listof=totoc,		% Abbildungs-/ Tabellenverzeichnis im Inhaltsverzeichnis darstellen
	toc=bibliography,	% Literaturverzeichnis im Inhaltsverzeichnis darstellen
]{scrreprt}	% scrreprt = Koma-Script report-Klasse, fuer laengere Bachelorarbeiten alternativ auch: scrbook

% Einstellungen laden
\usepackage{xstring}
\usepackage[utf8]{inputenc}
\usepackage[T1]{fontenc}

\usepackage{xargs} % mehrere Parameter angeben

%Variablen laden
\newcommand{\martrikelnr}{TODO MatrNR}
\newcommand{\titel}{TODO Titel}
\newcommand{\kurs}{TODO Kurs}
\newcommand{\datumAbgabe}{TODO Abgabe}
\newcommand{\firma}{TODO Firma}
\newcommand{\firmateilE}{TODO long name first}
\newcommand{\firmateilZ}{todo second half of long name}
\newcommand{\firmenort}{TODO Ort}
\newcommand{\abgabeort}{TODO Ort}
\newcommand{\abschluss}{TODO abschluss }
\newcommand{\studiengang}{TODO studiengang}
\newcommand{\institution}{TODO institution}
\newcommand{\dhbw}{TODO dhbw}
\newcommand{\betreuer}{TODO betreuer}
\newcommand{\gutachter}{TODO gutachter}
\newcommand{\zeitraum}{TODO zeitraum}
\newcommand{\arbeit}{TODO arbeit}
\newcommand{\semester}{TODO semester}
\newcommand{\autor}{TODO autor}
\newcommand{\schriftart}{helvet}
\newcommand{\seitenrand}{2.5cm}
\newcommand{\kapitelabstand}{0pt}
\newcommand{\spaltenabstand}{10pt}
\newcommand{\zeilenabstand}{1.5}
\newcommand{\zitierstil}{alphabetic}
\newcommand{\draftdisclaimer}{
    \begin{center} 
        \begin{framed} 
            \LARGE\textbf {DRAFT: DO NOT CITE OR DISTRIBUTE }
        \end{framed}
    \end{center}
}

% Einstellung der Sprache des Paketes Babel und der Verzeichnisüberschriften
\usepackage[english, ngerman]{babel}

%%%%%%% Package Includes %%%%%%%

\usepackage[margin=\seitenrand,foot=1cm]{geometry}	% Seitenränder und Abstände
\usepackage[activate]{microtype} %Zeilenumbruch und mehr
\usepackage[onehalfspacing]{setspace}
\usepackage{makeidx}
\usepackage[autostyle=true,german=quotes]{csquotes}
\usepackage{longtable}
\usepackage{enumitem}	% mehr Optionen bei Aufzählungen
\usepackage{graphicx}
\usepackage{xcolor} 	% für HTML-Notation
\usepackage{float}
\usepackage{array}
\usepackage{pifont}
\usepackage{calc}		% zum Rechnen (Bildtabelle in Deckblatt)
\usepackage[right]{eurosym}
\usepackage{wrapfig}	% zum einbinden von Bildern in den Fließtext
\usepackage{framed} 
\usepackage{pgffor} % für automatische Kapiteldateieinbindung
\usepackage[perpage, hang, multiple, stable]{footmisc} % Fussnoten
\usepackage[printonlyused]{acronym} % falls gewünscht kann die Option footnote eingefügt werden, dann wird die Erklärung nicht inline sondern in einer Fußnote dargestellt
\usepackage{listings}
\usepackage{pdfpages} % für einbinden von PDFs

%% Warnungen Filtern und auf relevante beschränken %%
\usepackage{scrhack}	% unterdrückt @float Warnungen
\usepackage{silence}	% unterdrückt spezifizierte Warnungen 
\WarningFilter{scrreprt}{Usage of package `fancyhdr'} % unterdrückt Warnung über Gebrauch von fancyhdr mit Koma-Script zusammen

%%% Hilfe beim schreiben %%%

% todo Notes
\usepackage[obeyFinal, colorinlistoftodos]{todonotes}
\newcommandx{\unsure}[2][1=]{\todo[linecolor=orange,backgroundcolor=orange!25,bordercolor=orange,#1]{#2}}
\newcommandx{\change}[2][1=]{\todo[linecolor=blue,backgroundcolor=blue!25,bordercolor=blue,#1]{#2}}
\newcommandx{\info}[2][1=]{\todo[linecolor=lime,backgroundcolor=lime!25,bordercolor=lime,#1]{#2}}
\newcommandx{\improvement}[2][1=]{\todo[linecolor=purple,backgroundcolor=purple!25,bordercolor=purple,#1]{#2}}
\newcommandx{\missing}[2][1=]{\todo[linecolor=yellow,backgroundcolor=yellow!25,bordercolor=yellow,#1]{#2}}

% für nutzung von Blindtext
\usepackage{blindtext}

%%%%%% Configuration %%%%%

%% Anwenden der Einstellungen

\usepackage[scaled]{\schriftart}
\renewcommand*\familydefault{\sfdefault}

\definecolor{LinkColor}{HTML}{00007A}
\definecolor{ListingBackground}{HTML}{E6E6E6}
\definecolor{gray}{rgb}{0.4,0.4,0.4}
\definecolor{darkblue}{rgb}{0.13,0.13,1}
\definecolor{cyan}{rgb}{0.0,0.6,0.6}
\definecolor{commentgreen}{rgb}{0.0,0.5,0}
\colorlet{punct}{magenta!60!black}
\definecolor{delim}{RGB}{20,105,176}
%\colorlet{numb}{magenta!60!black}

\usepackage{fancyhdr}
\fancypagestyle{plain}{%
	\fancyhf{}
	\fancyfoot[L]{\itshape\small\nouppercase{\leftmark}}
	\fancyfoot[R]{\thepage}
	\renewcommand{\headrulewidth}{0pt}
	\renewcommand{\footrulewidth}{0pt}
}
	
\fancypagestyle{chapter}{%
	\fancyhf{}
	\fancyfoot[R]{\thepage}
	\renewcommand{\headrulewidth}{0pt}
	\renewcommand{\footrulewidth}{0pt}
}

% Titel, Autor und Datum
\title{\titel}
\author{\autor}
\date{\datum}

% PDF Einstellungen
\usepackage[%
	pdftitle={\titel},
	pdfauthor={\autor},
	pdfsubject={\arbeit},
	pdfcreator={pdflatex, LaTeX with KOMA-Script},
	pdfpagemode=UseOutlines, 		% Beim Oeffnen Inhaltsverzeichnis anzeigen
	pdfdisplaydoctitle=true, 		% Dokumenttitel statt Dateiname anzeigen.
	pdflang={de}, 			% Sprache des Dokuments.
]{hyperref}
\usepackage[figure]{hypcap}

% (Farb-)einstellungen für die Links im PDF
\hypersetup{%
	colorlinks=true, 		% Aktivieren von farbigen Links im Dokument
	linkcolor=LinkColor, 	% Farbe festlegen
	citecolor=LinkColor,
	filecolor=LinkColor,
	menucolor=LinkColor,
	urlcolor=LinkColor,
	linktocpage=true, 		% Nicht der Text sondern die Seitenzahlen in Verzeichnissen klickbar
	bookmarksnumbered=true 	% Überschriftsnummerierung im PDF Inhalt anzeigen.
}
% Workaround um Fehler in Hyperref, muss hier stehen bleiben
\usepackage{bookmark} %nur ein latex-Durchlauf für die Aktualisierung von Verzeichnissen nötig

% Schriftart in Captions etwas kleiner
\addtokomafont{caption}{\small}
\renewcaptionname{ngerman}{\figurename}{Abb.}
\renewcaptionname{ngerman}{\tablename}{Tab.}

% Literaturverweise
\usepackage[
	backend=biber,		% empfohlen. Falls biber Probleme macht: bibtex
	bibwarn=true,
	bibencoding=utf8,	% wenn .bib in utf8, sonst ascii
	sortlocale=de-DE,
	style=\zitierstil,
]{biblatex}
\addbibresource{bibliographie.bib}


%%%%%% Additional settings %%%%%%

% Hurenkinder und Schusterjungen verhindern
% http://projekte.dante.de/DanteFAQ/Silbentrennung
\clubpenalty = 10000 % schließt Schusterjungen aus (Seitenumbruch nach der ersten Zeile eines neuen Absatzes)
\widowpenalty = 10000 % schließt Hurenkinder aus (die letzte Zeile eines Absatzes steht auf einer neuen Seite)
\displaywidowpenalty=10000

\tolerance 1414
\hbadness 1414
\emergencystretch 1.5em
\hfuzz 0.3pt
\vfuzz\hfuzz
\raggedbottom

% Bildpfad
\graphicspath{{images/}}

% Code Einstellungen
%\lstloadlanguages{[Sharp]C}
\lstdefinestyle{sharpc}{%
	language=[Sharp]C,		% Standardsprache des Quellcodes
	numbers=left,			% Zeilennummern links
	stepnumber=1,			% Jede Zeile nummerieren.
	numbersep=8pt,			% 5pt Abstand zum Quellcode
	numberstyle=\scriptsize,		% Zeichengrösse 'tiny' für die Nummern.
	breaklines=true,		% Zeilen umbrechen wenn notwendig.
	breakautoindent=true,	% Nach dem Zeilenumbruch Zeile einrücken.
	postbreak=\space,		% Bei Leerzeichen umbrechen.
	tabsize=1,				% Tabulatorgrösse 2
	basicstyle=\normalfont\ttfamily, % Nichtproportionale Schrift, klein für den Quellcode
	showspaces=false,		% Leerzeichen nicht anzeigen.
	showstringspaces=false,	% Leerzeichen auch in Strings ('') nicht anzeigen.
	extendedchars=true,		% Alle Zeichen vom Latin1 Zeichensatz anzeigen.
	captionpos=b,			% sets the caption-position to bottom
	backgroundcolor=\color{ListingBackground}, % Hintergrundfarbe des Quellcodes setzen.
	xleftmargin=0pt,		% Rand links
	xrightmargin=0pt,		% Rand rechts
	frame=single,			% Rahmen an
	frameround=ffff,
	rulecolor=\color{darkgray},	% Rahmenfarbe
	fillcolor=\color{ListingBackground},
	keywordstyle=\color{darkblue},
	commentstyle=\color{commentgreen},
	stringstyle=\color[rgb]{0.639,0.082,0.082},
}
\lstdefinelanguage{json}{
    basicstyle=\normalfont\ttfamily,
    string=[s][\color{black}]{"}{"},
    literate=
      {:}{{{\color{punct}{:}}}}{1}
      {,}{{{\color{punct}{,}}}}{1}
      {\{}{{{\color{delim}{\{}}}}{1}
      {\}}{{{\color{delim}{\}}}}}{1}
      {[}{{{\color{delim}{[}}}}{1}
      {]}{{{\color{delim}{]}}}}{1}
}
\lstdefinelanguage{XML}
{
  morestring=[b]",
  morestring=[s]{>}{<},
  morecomment=[s]{<?}{?>},
  stringstyle=\color{black},
  identifierstyle=\color{darkblue},
  keywordstyle=\color{cyan}
}

%define here some more styles
\lstset{style=sharpc}
\lstset{morekeywords={var}}
\lstset{literate=%
    {Ö}{{\"O}}1
    {Ä}{{\"A}}1
    {Ü}{{\"U}}1
    {ß}{{\ss}}1
    {ü}{{\"u}}1
    {ä}{{\"a}}1
    {ö}{{\"o}}1
    {~}{{\textasciitilde}}1
}

% Umbennung des Listings
\renewcommand\lstlistingname{Code}
\renewcommand\lstlistlistingname{Codeverzeichnis}
\def\lstlistingautorefname{Code}

% Abstände in Tabellen
\setlength{\tabcolsep}{\spaltenabstand}
\renewcommand{\arraystretch}{\zeilenabstand}

% Comfortänderungen
\newcommand{\fett}{\textbf}