% =======================================================
% NEUER ABSCHNITT: Spezielle Folien-Layouts
% =======================================================
\section{Spezielle Folien-Layouts}

\begin{frame}[plain]
    \frametitle{Folie ohne Kopf- und Fußzeile (plain)}
    
    Diese Folie wird mit der Option \texttt{[plain]} erstellt.
    Sie blendet die Kopf- und Fußzeile (Headline/Footline) des Themes 
    (z.B. Navigationsleiste, Autor, Datum) komplett aus.
    
    \vfill
    \begin{center}
    \Huge Ideal für eine reine Bild-Folie oder ein Zitat.
    \end{center}
    \vfill
    
    % Beachte: Der \frametitle wird technisch gesetzt,
    % aber vom 'plain' Style meistens auch ausgeblendet.
\end{frame}

\begin{frame}[plain]
    Diese Folie wird mit der Option \texttt{[plain]} erstellt.
    Sie blendet die Kopf- und Fußzeile (Headline/Footline) des Themes 
    (z.B. Navigationsleiste, Autor, Datum) komplett aus.
    
    \vfill
    \begin{center}
    \Huge Ohne Framtitle ist die Folie auch tatsächlich komplett leer.
    \end{center}
    \vfill
    
    % Beachte: Der \frametitle wird technisch gesetzt,
    % aber vom 'plain' Style meistens auch ausgeblendet.
\end{frame}

% --- Eine Folie nur ohne Kopfzeile (Headline) ---
{ % Beginne eine lokale LaTeX-Gruppe
    \setbeamertemplate{headline}{} % Leere die Kopfzeile nur für diese Folie
    \begin{frame}{Folie nur ohne Kopfzeile}
        Diese Folie hat keine Kopfzeile (den "Antibes"-Balken oben), 
        aber die normale Fußzeile (Footline) des Themes sollte noch sichtbar sein.
        \vfill
        Dies wurde durch lokales Setzen von 
        \texttt{\textbackslash setbeamertemplate\{headline\}\{\}}
        innerhalb von \texttt{\{\}} erreicht.
    \end{frame}
} % Die Gruppe endet, die Kopfzeile ist auf der nächsten Folie wieder da.