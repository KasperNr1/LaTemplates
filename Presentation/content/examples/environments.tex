% =======================================================
% NEUER ABSCHNITT: Weitere nützliche Elemente
% =======================================================
\section{Weitere nützliche Elemente}

\begin{frame}{Mathematische Formeln}
    \frametitle{Formeln mit \texttt{amsmath}}
    
    Eine Inline-Formel: $E = mc^2$.
    
    Eine abgesetzte, nummerierte Formel:
    \begin{equation}
        \int_0^\infty e^{-x^2} dx = \frac{\sqrt{\pi}}{2}
    \end{equation}
    
    Eine nicht-nummerierte Formel (mit \$\$...\$\$):
    $$ \sum_{n=1}^\infty \frac{1}{n^2} = \frac{\pi^2}{6} $$
\end{frame}

\begin{frame}[fragile] % [fragile] ist wichtig für 'verbatim' und 'listings'
    \frametitle{Code-Block (mit \texttt{listings})}
    
    Mit dem \texttt{listings}-Paket kann Code mit Syntax Highlighting
    dargestellt werden (Einstellungen siehe Präambel):
    
    \begin{lstlisting}[language=Python, caption={Ein Python-Beispiel}]
def hello_world():
    # Ein Kommentar
    print("Hello, Beamer!")
    
hello_world()
    \end{lstlisting}
\end{frame}

\begin{frame}{Tabellen}
    \frametitle{Eine Tabelle mit \texttt{booktabs}}
    
    \begin{table}
        \centering
        % \begin{adjustbox}{width=0.8\textwidth} % Falls Tabelle zu breit
        \begin{tabular}{lcr}
            \toprule
            Links ausgerichtet & Zentriert & Rechts ausgerichtet \\
            \midrule
            1.1 & 1.2 & 1.3 \\
            2.1 & 2.2 & 2.3 \\
            $\alpha$ & $\beta$ & $\gamma$ \\
            \bottomrule
        \end{tabular}
        % \end{adjustbox}
        \caption{Eine Beispieltabelle.}
    \end{table}
\end{frame}

