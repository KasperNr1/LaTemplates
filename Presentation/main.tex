\documentclass[12pt, aspectratio=169]{beamer}
\usepackage[utf8]{inputenc}
\usepackage[ngerman]{babel}

% --- Benötigte Pakete für die neuen Features ---
\usepackage{graphicx}      % Für Bilder (\includegraphics)
\usepackage{booktabs}      % Für schöne Tabellen (\toprule, \midrule, \bottomrule)
\usepackage{amsmath}       % Für erweiterte Mathe-Umgebungen (z.B. equation)
\usepackage{listings}      % Für Code-Blöcke
\usepackage{adjustbox}     % Zum Anpassen von Boxen (z.B. Tabellen)

% (Optional) Definiert einen Standardpfad für Bilder
% \graphicspath{{images/}} 

% Themes and Colors at https://www.overleaf.com/learn/latex/beamer#Reference_guide
\usetheme{Antibes}
\usecolortheme{beaver}

\setbeamertemplate{navigation symbols}{} % Hide PDF navigation symbols
\setbeamercovered{transparent} % Content
%\setbeamertemplate{headline}{} % No Navigation Bar at the Top 

% Fügt eine Fußzeile mit "X / Y" rechtsbündig ein
\setbeamertemplate{footline}{
  \hfill % Schiebt den Inhalt nach rechts
  \usebeamercolor{page number in head/foot}%
  \usebeamerfont{page number in head/foot}%
  \insertframenumber{} / \inserttotalframenumber
  \hspace*{2ex} % Kleiner Abstand zum rechten Rand
}

% --- Setup für das 'listings' Paket (optional) ---
\lstset{
    basicstyle=\ttfamily\small,
    keywordstyle=\color{blue},
    commentstyle=\color{gray},
    stringstyle=\color{red},
    breaklines=true,
    showstringspaces=false,
    frame=single, % Ein Rahmen um den Code
    captionpos=b, % Position der Caption (b=bottom)
    numbers=left, % Zeilennummern links
    numberstyle=\tiny\color{gray},
    language=Python % Standardsprache
}

\title{First Beamer Presentation}
\author{Matti Bos}
\date{\today}

% \logo{\includegraphics[height=1cm]{images/logoipsum.png}} % Add Logo on all Slides

\begin{document}

\thispagestyle{empty} % Remove Header and Footer on Title Page
\maketitle

\begin{frame}{Contents}
\tableofcontents
\end{frame}

\section{''Animating'' Bullet Points}
\begin{frame}{This is the heading}

\begin{itemize}
    \item <1-> From Start to end
    \item <2> Only slide 2
    \item <3-> Slide 3
\end{itemize}

Das hier ist Text zur Folie. Ich habe ihn nur einmal getippt und mit dynamischen Features ausgestattet. Zum Beispiel ist \alert<2>{das Hier auf Folie 2} farbig markiert.
\end{frame}

\begin{frame}[fragile]
\only<1>{Diese weitere Zeile ist nur auf der ersten Seite des Frames zu sehen}

\vspace{1em}

Der Text über mir braucht nur dann Platz, wenn er tatsächlich angezeigt wird. Das liegt an \verb|\only|. Normalerweise wird der Platz fest kalkuliert und leer gelassen.

\pause
Noch mehr Info

\end{frame}

\section{Boxes}
\begin{frame}
\frametitle{Sample frame title}

In this slide, some important text will be
\alert{highlighted} because it's important.
Please, don't abuse it.

\begin{block}{Remark}
Sample text
\end{block}

\begin{alertblock}{Important theorem}
Sample text in red box
\end{alertblock}

\begin{exampleblock}{Ein Deutsches Beispiel}
Sample text in green box. The title of the block is ``Examples".
\end{exampleblock}
\end{frame}

% =======================================================
% NEUER ABSCHNITT: Bilder und Spalten
% =======================================================
\section{Bilder und Spalten}

\begin{frame}{Einzelnes Bild}
    \frametitle{Ein Bild einfügen}
    Hier ist ein Bild. Es wird zentriert und auf 70% der Textbreite skaliert.
    
    \begin{figure}
        \centering
        \includegraphics[width=0.35\textwidth]{example-image-a}
        \caption{Dies ist eine Bildunterschrift.}
    \end{figure}
    
    % 'example-image-a' ist ein Platzhalterbild, das in den meisten
    % TeX-Distributionen enthalten ist. Ersetze es durch deinen eigenen Dateipfad.
\end{frame}

%\subsection{Eine Unterüberschrift}

\begin{frame}{Mehrere Bilder (via Spalten)}
    \frametitle{Zwei Bilder (oder Text \& Bild) nebeneinander}
    
    % [T] richtet die Spalten oben bündig aus (Top-aligned)
    % [c] wäre zentriert (Standard), [b] unten bündig
    \begin{columns}[T]
    
        % Linke Spalte
        \begin{column}{0.48\textwidth}
            \begin{figure}
                \includegraphics[width=\textwidth]{example-image-b}
                \caption{Linkes Bild}
            \end{figure}
        \end{column}
        
        % Rechte Spalte
        \begin{column}{0.48\textwidth}
            \begin{figure}
                \includegraphics[width=\textwidth]{example-image-c}
                \caption{Rechtes Bild}
            \end{figure}
        \end{column}
    \end{columns}
\end{frame}

\begin{frame}{Text und Bild in Spalten}
    \frametitle{Text neben einem Bild}
    
    \begin{columns}[c] % [c] = vertikal zentriert
        \begin{column}{0.6\textwidth}
            Hier steht der Text.
            \begin{itemize}
                \item Erklärung Punkt 1
                \item Erklärung Punkt 2
                \item Das Bild rechts illustriert diese Punkte auf hervorragende Weise.
            \end{itemize}
        \end{column}
        
        \begin{column}{0.4\textwidth}
            \begin{figure}
                \includegraphics[width=\textwidth]{example-image}
            \end{figure}
        \end{column}
    \end{columns}
\end{frame}

% =======================================================
% NEUER ABSCHNITT: Spezielle Folien-Layouts
% =======================================================
\section{Spezielle Folien-Layouts}

\begin{frame}[plain]
    \frametitle{Folie ohne Kopf- und Fußzeile (plain)}
    
    Diese Folie wird mit der Option \texttt{[plain]} erstellt.
    Sie blendet die Kopf- und Fußzeile (Headline/Footline) des Themes 
    (z.B. Navigationsleiste, Autor, Datum) komplett aus.
    
    \vfill
    \begin{center}
    \Huge Ideal für eine reine Bild-Folie oder ein Zitat.
    \end{center}
    \vfill
    
    % Beachte: Der \frametitle wird technisch gesetzt,
    % aber vom 'plain' Style meistens auch ausgeblendet.
\end{frame}

\begin{frame}[plain]
    Diese Folie wird mit der Option \texttt{[plain]} erstellt.
    Sie blendet die Kopf- und Fußzeile (Headline/Footline) des Themes 
    (z.B. Navigationsleiste, Autor, Datum) komplett aus.
    
    \vfill
    \begin{center}
    \Huge Ohne Framtitle ist die Folie auch tatsächlich komplett leer.
    \end{center}
    \vfill
    
    % Beachte: Der \frametitle wird technisch gesetzt,
    % aber vom 'plain' Style meistens auch ausgeblendet.
\end{frame}

% --- Eine Folie nur ohne Kopfzeile (Headline) ---
{ % Beginne eine lokale LaTeX-Gruppe
    \setbeamertemplate{headline}{} % Leere die Kopfzeile nur für diese Folie
    \begin{frame}{Folie nur ohne Kopfzeile}
        Diese Folie hat keine Kopfzeile (den "Antibes"-Balken oben), 
        aber die normale Fußzeile (Footline) des Themes sollte noch sichtbar sein.
        \vfill
        Dies wurde durch lokales Setzen von 
        \texttt{\textbackslash setbeamertemplate\{headline\}\{\}}
        innerhalb von \texttt{\{\}} erreicht.
    \end{frame}
} % Die Gruppe endet, die Kopfzeile ist auf der nächsten Folie wieder da.

% =======================================================
% NEUER ABSCHNITT: Weitere nützliche Elemente
% =======================================================
\section{Weitere nützliche Elemente}

\begin{frame}{Mathematische Formeln}
    \frametitle{Formeln mit \texttt{amsmath}}
    
    Eine Inline-Formel: $E = mc^2$.
    
    Eine abgesetzte, nummerierte Formel:
    \begin{equation}
        \int_0^\infty e^{-x^2} dx = \frac{\sqrt{\pi}}{2}
    \end{equation}
    
    Eine nicht-nummerierte Formel (mit \$\$...\$\$):
    $$ \sum_{n=1}^\infty \frac{1}{n^2} = \frac{\pi^2}{6} $$
\end{frame}

\begin{frame}[fragile] % [fragile] ist wichtig für 'verbatim' und 'listings'
    \frametitle{Code-Block (mit \texttt{listings})}
    
    Mit dem \texttt{listings}-Paket kann Code mit Syntax Highlighting
    dargestellt werden (Einstellungen siehe Präambel):
    
    \begin{lstlisting}[language=Python, caption={Ein Python-Beispiel}]
def hello_world():
    # Ein Kommentar
    print("Hello, Beamer!")
    
hello_world()
    \end{lstlisting}
\end{frame}

\begin{frame}{Tabellen}
    \frametitle{Eine Tabelle mit \texttt{booktabs}}
    
    \begin{table}
        \centering
        % \begin{adjustbox}{width=0.8\textwidth} % Falls Tabelle zu breit
        \begin{tabular}{lcr}
            \toprule
            Links ausgerichtet & Zentriert & Rechts ausgerichtet \\
            \midrule
            1.1 & 1.2 & 1.3 \\
            2.1 & 2.2 & 2.3 \\
            $\alpha$ & $\beta$ & $\gamma$ \\
            \bottomrule
        \end{tabular}
        % \end{adjustbox}
        \caption{Eine Beispieltabelle.}
    \end{table}
\end{frame}

% =======================================================
% NEUER ABSCHNITT: Referenzen
% =======================================================
\section{Referenzen}

\begin{frame}[allowframebreaks] % Erlaube Seitenumbrüche, falls die Liste lang ist
    \frametitle{Literaturverzeichnis}
    
    Man kann Beamer-Folien zitieren \cite{Knuth97}.
    Oder auch andere Quellen \cite{Lamport94}.
    
    \vspace{1em}
    
    % Einfache 'thebibliography' Umgebung, funktioniert ohne BibTeX/BibLaTeX
    \begin{thebibliography}{99} % {99} reserviert Platz für zweistellige Nummern
    
        \bibitem{Knuth97}
        Donald E. Knuth (1997).
        \textit{The Art of Computer Programming, Volume 1: Fundamental Algorithms}.
        Addison-Wesley Professional.
        
        \bibitem{Lamport94}
        Leslie Lamport (1994).
        \textit{\LaTeX: A Document Preparation System}.
        Addison-Wesley Professional.
    
    \end{thebibliography}
\end{frame}

% --- Leere Frames vom Original ---
\section{More Content}
\begin{frame}{Very interesting Title}
passiert hier auch noch irgendwas?
    
\end{frame}

\begin{frame}
    \frametitle{Fake}
    
\end{frame}

\end{document}