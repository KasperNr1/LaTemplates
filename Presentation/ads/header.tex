\documentclass[12pt, aspectratio=169]{beamer}
\usepackage[utf8]{inputenc}
\usepackage[ngerman]{babel}

% --- Benötigte Pakete für die neuen Features ---
\usepackage{graphicx}      % Für Bilder (\includegraphics)
\usepackage{booktabs}      % Für schöne Tabellen (\toprule, \midrule, \bottomrule)
\usepackage{amsmath}       % Für erweiterte Mathe-Umgebungen (z.B. equation)
\usepackage{listings}      % Für Code-Blöcke
\usepackage{adjustbox}     % Zum Anpassen von Boxen (z.B. Tabellen)

% (Optional) Definiert einen Standardpfad für Bilder
% \graphicspath{{images/}} 

% Themes and Colors at https://www.overleaf.com/learn/latex/beamer#Reference_guide
\usetheme{Antibes}
\usecolortheme{beaver}

\setbeamertemplate{navigation symbols}{} % Hide PDF navigation symbols
\setbeamercovered{transparent} % Content
%\setbeamertemplate{headline}{} % No Navigation Bar at the Top 

% Fügt eine Fußzeile mit "X / Y" rechtsbündig ein
\setbeamertemplate{footline}{
  \hfill % Schiebt den Inhalt nach rechts
  \usebeamercolor{page number in head/foot}%
  \usebeamerfont{page number in head/foot}%
  \insertframenumber{} / \inserttotalframenumber
  \hspace*{2ex} % Kleiner Abstand zum rechten Rand
}

% --- Setup für das 'listings' Paket (optional) ---
\lstset{
    basicstyle=\ttfamily\small,
    keywordstyle=\color{blue},
    commentstyle=\color{gray},
    stringstyle=\color{red},
    breaklines=true,
    showstringspaces=false,
    frame=single, % Ein Rahmen um den Code
    captionpos=b, % Position der Caption (b=bottom)
    numbers=left, % Zeilennummern links
    numberstyle=\tiny\color{gray},
    language=Python % Standardsprache
}

\title{First Beamer Presentation}
\author{Matti Bos}
\date{\today}
